
\section{Introduction}
Digital advertising has always been the financial backbone of the web. A lot of advertising networks facilitate advertising on the web. While the advertising market is dominated by a few large ad network such as Google, Facebook and Media.net (powered by Yahoo) there are a myriad of smaller ad networks that also play an important role in the web ecosystem. Of late, these smaller ad networks have been facing increasingly adverse conditions. Apart from the intense pressure of having to compete with large media companies such as Google, Facebook and Yahoo, these ad networks are having to contend with an increasing number of users using various kinds of ad blockers. Ad blocking browser extensions (such as AdBlock and ABP), browsers built with ad blocking capabilities (such as Brave and Chrome) are all producing an adverse impact on the revenue of these smaller ad networks. As a result, these ad networks are trying to pursue alternate avenues to keep up their revenues. One such avenue is the Web Push technology implemented by modern browsers. Web Push technology allows for web content developers to send out Web Push Notifications (WPN), which re-engage their visitors. Unlike mobile app-based push notifications, WPNs allow for notifications to be displayed on both desktop and mobile devices alike. Thus, they are a great tool for quality content developers to re-connect to their past users.  
\roberto{The above paragraph is still verbose and somewhat repetitive. I'll try to do another pass later to remove redundancy}

While WPNs have applicability as a great tool for content developers, several ad networks have begun misusing them for a different purpose: as an ad-delivery mechanism. 
%
\roberto{Is this really a misuse? It seems that legitimate ads may actually be one of the use cases intended for notifications, no?}
%
In effect, the notifications that pop up on user devices take the place of the ads on publisher websites. The user, instead of clicking on an ad, will need to click on a notification message which will then lead the user to a landing page of a marketer. The marketer will then pay the ad network for that particular user visit. Some of that money will also trickle down to the publisher who was responsible for this push notification subscription in the first place. The usage of WPNs for ad delivery has a couple of unique advantages. Firstly, the chances of users engaging with the advertised content is higher than other traditional ad-delivery mechanisms such as banner ads, pop-up ads or pop-under ads. In our preliminary research on this topic, we have come across discussion among publishers and ad-network managers about the effectiveness of push ads being much higher than other mechanisms ~\cite{x}. This could presumably be attributed to the fact that in traditional ads the ad space often will have to compete with the content space for users attention. On the other hand, through years of training with the help of mobile app notifications, users have been trained to compulsively interact with push notification messages (at least on mobile devices). Further, WPN-based ads are less prone to phenomena such as "ad-blindness" effects~\cite{x}, as are some other ad delivery mechanisms such as page banners. These factors could contribute to the higher user engagement metrics for WPN-based ads. Secondly, in WPN-based ads users click on the notification and directly navigate to the landing page. Hence, none of the existing ad-blocking mechanism (whether built into the browser or extensions) have an easy way to stop this navigation as the user is herself directly clicking on the notification. \roberto{Is it true that ad-blocking is difficult? Can't the adblock happen before ad registration? Can they block the service worker code?}. This also contributes to the effectiveness of WPN-based ads. This is even more important in the light of new steps being taken by the developers of the highly popular Chrome browser to stop intrusive ads~\cite{x}. 


\roberto{I feel that there is too much background on WPN ads in general, but little information/examples of abuse}
%
In order to use the WPNs as an effective ad-delivery mechanism, the notification still has to be convincing enough to elicit a click (or a tap) from the user. Due to the general reluctance of the users in viewing advertisements, WPN-based ads often tend to use misleading notification messages for their ads. This problem is further exacerbated by the presence of affiliate marketers who act as middle-men between the ad-networks and the advertisers~\cite{}. These affiliate marketers try to create campaigns whose sole job is entice the users by any means to click on the notification messages and do a follow-up action. These actions can be as simple as submitting an Email-address and name information or downloading a file or eliciting credit card information from a user (such as subscribing to a service). The affiliate marketers are usually paid on a Cost-Per-Action (CPA) basis. This use of misleading messages and aggressive middle-men affiliate marketers thus has a potential to drive good marketers away and attract malicious advertisers. The goal of of our research is to systematically study the push advertising ecosystem and determine if this is indeed the case. To this end, we built a system called PushAdMiner. PushAdMiner is a scalable framework that can subscribe to WPN-based ads on thousands of different websites delivered through various ad networks as determined by an initial seed list. The framework can then receive simultaneously receive the WPN-based ads from all its subscriptions in a parallel manner. Each time a WPN arrives, the system records all the details of the WPN and then interacts with it in order to capture the landing page as well. Along with screenshots, detailed forensic logs of all publsiher websites, background WPN-related scripts, landing pages are all recorded by PushAdMiner. PushAdMiner is designed for both desktop as well as mobile platforms. After developing PushAdMiner, we used it visit x websites on mobile and desktop platforms and subscribed to x WPN ad services. In total, we received and interacted with x notification messages. By using various clustering techniques, we categorized both the WPN messages as well as the landing pages. We noticed that the landing pages of WPN-based ads included a large number of various scams such as survey scam websites, fake software download sites and technical support scams.

To the best of our knowledge, this is the first systematic security-based research study on the ecosystem of push advertising. Previous research work has shown that Push notification services have a potential to be abused for building stealthy botnets~\cite{master_web_puppets} but the push advertising ecosystem has not been studied.  Previous work has also alluded to some malvertisements trying to social engineer users into subscribing to Push ads~\cite{imc19_paper} but this work did not go into the details of what would happen after the users actually subscribe to these push services. Our research work is the first to systematically study WPN-based ads in an automated manner by subscribing to and interacting with these ads. These are the list of contributions:
\begin{enumerate}
    \item
\end{enumerate}